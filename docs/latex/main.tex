\documentclass{article}
\usepackage{graphicx}
\usepackage{amsmath}
\usepackage[
  a4paper, mag=1000,
  left=3.0cm, right=1cm, top=2cm, bottom=2cm, headsep=0.7cm, footskip=1cm
]{geometry}
\DeclareMathOperator{\sech}{sech}

\title{RRSPH v3.0.20}
\author{Asrankulov Alexander}
\date{October 2024}

\begin{document}

\maketitle

\section{Experiment}

\subsection{Experiment directory}\

Experiment directory is a directory with parameters files and particles data. 
You should provide specifically prepared directory to solver.

Minimal experiment directory to start an experiment is presented below:
\begin{verbatim}
\---experiment_00
    |   ModelParams.json
    |   ParticleParams.json
    |   
    \---dump
            0.csv
\end{verbatim}

Any RRSPH simulation program started from parent directory will produce the following output:
\begin{verbatim}
Found experiments:
[-1] Change search directory
[0] experiment_00: (0/1) data/dump layers
Type experiment id you want to load:
>
\end{verbatim}
You should type '0' and press 'Enter' in order to start simulation with specified parameters and initial state.

Experiment directories with 'ParticleParams.json' and initial state ('dump/0.csv') are usually generated with scripts. Mathematical model's 'ModelParams.json' is user-provided: manually or by SPH2DParamsGenerator app.

\subsection{RRSPH package}

\subsubsection{SPH2D simulation}\

There are two simulation programs:
\begin{itemize}
    \item SPH2D\_OMP. SPH 2D solver running on CPU with OpenMP or single-threaded.
    \item SPH2D\_CL. SPH 2D solver running on GPU with OpenCL runtime. It's executable file is distributed with CL source code in separate directory:

    \begin{verbatim}
        |   SPH2D_CL.exe
        |
        +---cl
        |       ArtificialViscosity.cl
        |       AverageVelocity.cl
        |       clErrorCodes.txt
        |       clparams.h
        |       common.h
        |       Density.cl
        |       EOS.cl
        |       ExternalForce.cl
        |       GridFind.cl
        |       GridUtils.h
        |       InternalForce.cl
        |       ParamsEnumeration.h
        |       SmoothingKernel.cl
        |       SmoothingKernel.h
        |       TimeIntegration.cl
        |   
        \---experiment_00
            |   ModelParams.json
            |   ParticleParams.json
            |   SPH2DParams.json
            |   
            \---dump
                    0.csv
    \end{verbatim}
\end{itemize}

\subsubsection{SPH2D Post-processing}

There's several post-processing programs:
\begin{itemize}
    \item WaterProfile. This app can be used to compute water profiles in space and time.
    \item FuncAtPoint. This app can be used to extract field variable at specified point in space by SPH smoothing.
    \item PartToGridConverter. This app can be used to convert SPH particles output into grid to process as mesh algorithm's output.
\end{itemize}

\section{Parameters Scheme}

\subsection{Model Params}

\subsubsection{Target version}

\begin{itemize}
    \item \verb|params_target_version_major|
    \item \verb|params_target_version_minor|
    \item \verb|params_target_version_patch|
\end{itemize}

\subsubsection{Smoothing kernel function}
\label{sec:skf}

Possible values:

\begin{itemize}
    \item 1: \verb|SKF_CUBIC|.
    \begin{equation}
        W(q) = \kappa_{\text{cubic}}
        \begin{cases}
          \begin{tabular}{l l}
              $\frac{2}{3} - q^{2} + \frac{1}{2} q^{3}$
              &
              $q \leq 1$, 
              
              \\
              
              $\frac{1}{6} \left(2 - q\right)^{3}$
              &
              $1 < q \leq 2$,
              \\
              
              0
              &
              $q > 2$.
          \end{tabular}  
        \end{cases} 
    \end{equation}

    \item 2: \verb|SKF_GAUSS|.
    \begin{equation}
        W(q) = \kappa_{\text{gauss}}
        \begin{cases}
          \begin{tabular}{l l}
              $\displaystyle{\frac{1}{h^{2}\pi}e^{q^{-2}}}$
              &
              $q \leq 3$, 
              
              \\
              
              0
              &
              $q > 3$.
          \end{tabular}  
        \end{cases} 
    \end{equation}

    \item 3: \verb|SKF_WENDLAND|.
    \begin{equation}
        W(q) = \kappa_{\text{wendland}}
        \begin{cases}
          \begin{tabular}{l l}
              $\displaystyle{\left(
              1 - \frac{q}{2}
              \right)^4
              (2q + 1)}$
              &
              $q \leq 2$, 
              
              \\
              
              0
              &
              $q > 2$.
          \end{tabular}  
        \end{cases} 
    \end{equation}
    
    \item 3: \verb|SKF_DESBRUN|.
    \begin{equation}
        W(q) = \kappa_{\text{desbrun}}
        \begin{cases}
          \begin{tabular}{l l}
              $\displaystyle{\left(2 - q
              \right)^3}$
              &
              $q \leq 2$, 
              
              \\
              
              0
              &
              $q > 2$.
          \end{tabular}  
        \end{cases} 
    \end{equation}
\end{itemize}


\subsubsection{Density}

\begin{itemize}
    \item \verb|density_treatment|. Density integration function. Possible values of enumeration:
    \begin{itemize}
        \item 0: \verb|DENSITY_SUMMATION|. Use SPH summation over kernel:
        \begin{equation}
            \rho_{j} = \sum\limits_{i} m_{i} W_{ji}.
        \end{equation}

        \item 1: \verb|DENSITY_CONTINUITY|. Use integration of continuity equation:
        \begin{equation}
            \frac{D\rho_{j}}{Dt} = \sum\limits_{i} m_{i} \vec{v}_{ji} \cdot \vec{\nabla}_{j} W_{ji}.
        \end{equation}

        \item 2: \verb|DENSITY_CONTINUITY_DELTA|. Use integration of continuity equation with density diffussion:
        \begin{equation}\label{eq:CSPM}
             \frac{D\rho_{j}}{Dt} = \sum\limits_{i} m_{i} \vec{v}_{ji} \cdot \vec{\nabla}_{j} W_{ji} + 
             2\delta_{\text{density}} h c
             \sum\limits_{i}m_{i} \frac{(\rho_{i} - \rho_{j})}{\rho_{i}} \frac{\vec{r}_{ji}}{|\vec{r}_{ji}|^2} \cdot \vec{\nabla}_{j} W_{ji}.
        \end{equation}
    \end{itemize}

    \item \verb|density_normalization|. Enable boundary deficiency correction:
    \begin{equation}
        \rho_{j} = \frac{
        \sum\limits_{i}m_{i}W_{ji}
        }{
        \sum\limits_{i}
        \left(\frac{m_{i}}{\rho_{i}}\right) W_{ji}
        }
    \end{equation}

    Optional. Available if \verb|density_treatment == 0|.
    \item \verb|density_skf|. See section~\ref{sec:skf}.
    \item \verb|density_delta_sph_coef|. $\delta_{\text{density}}$ parameter in eq.~\ref{eq:CSPM}.
    
    Mandatory if \verb|density_treatment == 2|.
\end{itemize}

\subsubsection{Equation of state}

In order to simulate water artificial equation of state is used:
\begin{equation}
    p = \frac{c_{0}^{2}\rho_{0}}{\gamma}
    \left[
    \left(
    \frac{\rho}{\rho_{0}}
    \right) ^ {\gamma}
    -1
    \right].
\end{equation}

\begin{itemize}
    \item \verb|eos_sound_vel_method|. Method for sound velocity choosing. Possible values of enumeration:

    \begin{itemize}
        \item 0: \verb|EOS_SOUND_VEL_DAM_BREAK|. Use dam break assumption:
        \begin{equation}\label{eq:cdam}
            c = \sqrt{200gdc_{k}}.
        \end{equation}

        \item 1: \verb|EOS_SOUND_VEL_SPECIFIC|. Use custom sound velocity.
        \begin{equation}\label{eq:ccustom}
            c = c_{\text{user}}.
        \end{equation}
    \end{itemize}

    \item  \verb|eos_sound_vel_coef|. $c_{k}$ parameter in eq.~\ref{eq:cdam}.

    Mandatory if \verb|eos_sound_vel_method == 0|.
    
    \item \verb|eos_sound_vel|. User-provided sound velocity $c_{\text{user}}$ in eq.~\ref{eq:ccustom}.
    
    Mandatory if \verb|eos_sound_vel_method == 1|.
\end{itemize}

\subsubsection{Internal Forces}

\begin{itemize}
    \item \verb|intf_sph_approximation|. SPH momentum equation form. Possible values of enumeration:
    \begin{itemize}
        \item 1: \verb|INTF_SPH_APPROXIMATION_1|. SPH momentum equation form:
        \begin{equation}
            \frac{D\vec{v}_j}{D t} = 
            -\sum\limits_{i} m_{i} 
            \left(
            \frac{p_{i} + p_{j}}{\rho_{i} \rho_{j}}
            \right) \vec{\nabla}_{j} W_{ji}.
        \end{equation}
        
        \item 2: \verb|INTF_SPH_APPROXIMATION_2|. SPH momentum equation form:
        \begin{equation}
            \frac{D\vec{v}_j}{D t} = 
            -\sum\limits_{i} m_{i} 
            \left(
            \frac{p_{j}}{\rho_{j}^{2}} + 
            \frac{p_{i}}{\rho_{i}^{2}}
            \right) \vec{\nabla}_{j} W_{ji}.
        \end{equation}
    \end{itemize}

    \item \verb|intf_hsml_coef|. $h_{k}$ in smoothing kernel length equation:
    \begin{equation}
        h = \delta_{0} \cdot h_{k},
    \end{equation}
    where $\delta_{0}$ is initial distance between particles.

    \item \verb|intf_skf|. See section~\ref{sec:skf}.
\end{itemize}

\subsubsection{Artificial Pressure}

\begin{itemize}
    \item \verb|artificial_pressure|. Enable additional factor in momentum equation for tensile instability correction:
    \begin{equation}\label{eq:artpress0}
        \frac{D\vec{v}_j}{D t} = 
        -\sum\limits_{i} m_{i} 
        \left(
        \frac{p_{j}}{\rho_{j}^{2}} + 
        \frac{p_{i}}{\rho_{i}^{2}} +
        R\Tilde{W}_{ji}^{n}
        \right) \vec{\nabla}_{j} W_{ji},
    \end{equation}
    where $R\Tilde{W}_{ji}^{n}$ is artificial pressure term. SPH approximation function could be of both types.
    
    \begin{equation}
        R = R_{i} + R_{j},
    \end{equation}

    where any indexed $R_{x}$ is:
    \begin{equation}\label{eq:artpress1}
        R_{x} = 
        \begin{cases}
          \begin{tabular}{l l}
              $\displaystyle{
              \frac{\epsilon|p_{x}|}{\rho_{x}^{2}}
              }$
              &
              $p_{x} < 0$, 
              
              \\
              
              0
              &
              $p_{x} \geq 0$,
          \end{tabular}  
        \end{cases} 
    \end{equation}
    where $\epsilon$ is artificial pressure coefficient.
    
    \begin{equation}
        \Tilde{W}_{ji} = \frac{W(\vec{r}_{ji}, h)}{W(\delta_{0}, h)},
    \end{equation}
    where $\delta_{0}$ is initial distance between particles.

    \item \verb|artificial_pressure_skf|. See section \ref{sec:skf}.
    \item \verb|artificial_pressure_index|. Term $n$ in eq.~\ref{eq:artpress0}.
    \item \verb|artificial_pressure_coef|. Term $\epsilon$ in eq.~\ref{eq:artpress1}.
\end{itemize}

\subsubsection{Artificial Viscosity}
\begin{itemize}
    \item \verb|artificial_viscosity|. Enable additional factor in momentum equation for stability correction:
    \begin{equation}
        \frac{D\vec{v}_j}{D t} = 
        -\sum\limits_{i} m_{i} 
        \left(
        \frac{p_{j}}{\rho_{j}^{2}} + 
        \frac{p_{i}}{\rho_{i}^{2}} +
        \Pi_{ji}
        \right) \vec{\nabla}_{j} W_{ji},
    \end{equation}
    where $\Pi_{ji}$ is artificial viscosity term. SPH approximation function could be of both types.
    $\Pi_{ji}$ is:
    \begin{equation}\label{eq:artvisc}
        \Pi_{ji} = 
        \begin{cases}
          \begin{tabular}{l l}
              $\displaystyle{
              \frac{-\alpha_{\Pi}c_{ji}\phi_{ji} + \beta_{\Pi}\phi_{ji}^{2}}{\rho_{ji}^{2}}
              }$
              &
              $\vec{v}_{ji} \cdot \vec{r}_{ji} < 0$, 
              
              \\
              
              0
              &
              $\vec{v}_{ji} \cdot \vec{r}_{ji} \geq 0$,
          \end{tabular}  
        \end{cases} 
    \end{equation}

    \begin{equation}
        \phi_{ji} = 
        \frac
        {h_{ji}\vec{v}_{ji} \cdot \vec{r}_{ji}}
        {|\vec{r}_{ji}|^2 + (\epsilon_{\Pi} h_{ji})^2 },
    \end{equation}

    \begin{equation}
        \begin{tabular}{l l l}
              $\displaystyle{
              \rho_{ji} = \frac{\rho_{j} + \rho_{i}}{2}
              }$,
              &
              $\displaystyle{
              c_{ji} = \frac{c_{j} + c_{i}}{2}
              }$, 
              &
              $\displaystyle{
              h_{ji} = \frac{h_{j} + h_{i}}{2}
              }$,
          \end{tabular}  
    \end{equation}
    
    \begin{equation}
        \begin{tabular}{l l}
              $\displaystyle{
              \vec{v}_{ji} = \vec{v}_{j} - \vec{v}_{i}
              }$,
              &
              $\displaystyle{
              \vec{r}_{ji} = \vec{r}_{j} - \vec{r}_{i}
              }$.
          \end{tabular}  
    \end{equation}

    \item \verb|artificial_viscosity_skf|. See section~\ref{sec:skf}.
    \item \verb|artificial_shear_visc|. Term $\alpha_{\Pi}$ in eq.~\ref{eq:artvisc}.
    \item \verb|artificial_bulk_visc|. Term $\beta_{\Pi}$ in eq.~\ref{eq:artvisc}.
\end{itemize}

\subsubsection{Average Velocity}
\begin{itemize}
    \item \verb|average_velocity|. Enable XSPH velocity smoothing factor:
    \begin{equation}
        \frac{D\vec{r}_j}{Dt} = \vec{v}_{j} + \vec{v}_{j\;\text{avg}},
    \end{equation}
    where $v_{avg}$ is:
    \begin{equation}\label{eq:xsph}
        \vec{v}_{j\;\text{avg}} = -\varepsilon_{\text{avg}}
        \sum\limits_{i}\frac{m_{i}}{\rho_{i}}\vec{v}_{ji}W_{ji}.
    \end{equation}

    \item \verb|average_velocity_skf|. See section~\ref{sec:skf}.
    \item \verb|average_velocity_coef|. Term $\varepsilon_{\text{avg}}$ in eq.~\ref{eq:xsph}.
\end{itemize}

\subsubsection{Time Integration}
\begin{itemize}
    \item \verb|simulation_time|. Total simulation time to stop experiment.
    \item \verb|dt_correction_method|. Method of $\Delta t$ choosing. Possible values of enumeration:
    \begin{itemize}
        \item 0: \verb|DT_CORRECTION_CONST_VALUE|. Use user-provided constant value.
        \begin{equation}\label{eq:dtcvalue}
            \Delta t_{\tau} = \Delta t_{\text{user}}.
        \end{equation}
        \item 1: \verb|DT_CORRECTION_CONST_CFL|. $\Delta t$ is computed before experiment start with CFL condition:
        \begin{equation}\label{eq:dtccfl}
            \Delta t_{\tau} = \Delta t_{0} = 
            \text{CFL} \frac{h}{c (1 + 1.2\alpha_{\Pi})}
        \end{equation}
        \item 2: \verb|DT_CORRECTION_DYNAMIC|. $\Delta t$ is computed on every step with CFL condition:
        \begin{equation}\label{eq:dtdcfl}
            \Delta t_{\tau} = \text{CFL}\min(\Delta t_{f}, \Delta t_{\phi}),
        \end{equation}
        \begin{equation}
            \Delta t_{f} = \min\limits_{i}
            \left(
            \sqrt{\frac{h}{a_{i}}}
            \right),
        \end{equation}
        \begin{equation}
            \Delta t_{\phi} = \min\limits_{i}
            \left(
            \frac{h}{c + \phi}
            \right).
        \end{equation}
    \end{itemize}

    \item \verb|dt|. Term $\Delta t_{\text{user}}$ in eq.~\ref{eq:dtcvalue}.
    \item \verb|CFL_coef|. Term CFL in eq.~\ref{eq:dtccfl} and eq.~\ref{eq:dtdcfl}.
\end{itemize}

\subsubsection{Boundary Treatment}
\begin{itemize}
    \item \verb|boundary_treatment|. Selects method for boundaries. Possible values of enumeration:
    \begin{itemize}
        \item 0: \verb|SBT_DYNAMIC|. Boundary particles are the same as fluid but have constant positions.
        \item 1: \verb|SBT_REPULSIVE|. Boundary particles are dynamic with additional term in momentum equation:
        
        \begin{equation}
        \frac{D\vec{v}_j}{D t} = 
        -\sum\limits_{i} 
        \left[
        m_{i} 
        \left(
        \frac{p_{j}}{\rho_{j}^{2}} + 
        \frac{p_{i}}{\rho_{i}^{2}}
        \right) \vec{\nabla}_{j} W_{ji}
        + \vec{\Upsilon}_{ji}
        \right],
        \end{equation}
        
        where $\vec{\Upsilon}_{ji}$ is computed if $j$ is fluid particle and $i$ is boundary particle:
        
        \begin{equation}
        \vec{\Upsilon}_{ji} = 
        \frac{\vec{r}_{ji}}{|\vec{r}_{ji}|}
        \begin{cases}
          \begin{tabular}{l l}
              $\displaystyle{
              D\left[
              \left(
                \frac{r_{0}}{|\vec{r}_{ji}|}
              \right)^{\alpha_{1}}
              -
              \left(
                \frac{r_{0}}{|\vec{r}_{ji}|}
              \right)^{\alpha_{2}}
              \right]
              }$
              &
              $|\vec{r}_{ji}| \leq r_{0}$, 
              
              \\
              
              0
              &
              $|\vec{r}_{ji}| > r_{0}$.
          \end{tabular}  
        \end{cases} 
        \end{equation}
        Equation parameters are considered built-in:
        \begin{equation}
          \begin{tabular}{l}
            $r_{0} = 2h$,
            \\
            $D = 5gd$,
            \\
            $\alpha_{1} = 12$,
            \\
            $\alpha_{2} = 4$,
          \end{tabular}  
        \end{equation}
        where $d$ is initial water depth.
    \end{itemize}
\end{itemize}

\subsubsection{Numerical Waves Maker}
\begin{itemize}
    \item \verb|nwm|. Waves maker method. Possible values of enumeration:
    \begin{itemize}
        \item 0: \verb|NWM_NO_WAVES|. No waves generator.
        \item 2: \verb|NWM_METHOD_DYNAMIC_1|. First order waves generator. Piston-type wavemaker. 

        Makes waves with surface displacement of type:
        \begin{equation}
            \eta(x, t) = \frac{H}{2}cos(\omega t - kx + \delta),
        \end{equation}
        where $H$ is wave height, $x$ is distance and $\delta$ is the initial phase.
        $\omega = 2\pi/T$ is the angular frequency and $k=2\pi/L$ is the wave number with $T$ equal to the wave period and $L$ the wave length.

        Piston displacement equation is:
        \begin{equation}\label{eq:NWMD1}
            e(t) = e_{1}(t) = \frac{S_{0}}{2} sin(\omega t + \delta),
        \end{equation}
        where $S_{0}$ is piston magnitude:
        \begin{equation}
            S_{0} = \frac{H}{m_{1}},
        \end{equation}
        where $m_{1}$:
        \begin{equation}
            m_{1} = \frac
            {2\sinh^{2}(kd)}
            {\sinh(kd)\cosh(kd) + kd},
        \end{equation}
        where $d$ is depth.

        \item 3: \verb|NWM_METHOD_DYNAMIC_2|. Second order waves generator. Piston-type wavemaker.

        Piston displacement equation extends eq.~\ref{eq:NWMD1} with second-order term:
        \begin{equation}
            e(t) = e_{1}(t) + e_{2}(t),
        \end{equation}
        where $e_{2}$ is:
        \begin{equation}
            e_{2}(t) = \left[
            \left(
            \frac{H^{2}}{32d}
            \right)
            \cdot
            \left(
            \frac{3\cosh(kd)}{\sinh^{3}(kd)} - \frac{2}{m_{1}}
            \right)
            \right]
            \sin(2\omega t + 2\delta).
        \end{equation}
        
        \item 4: \verb|NWM_METHOD_WALL_DISAPPEAR|.
        \item 5: \verb|NWM_METHOD_SOLITARY_RAYLEIGH|.

        Solitary wave generator. 
        Piston-type wavemaker. 
        Can be used to generate only one solitary wave.

        Piston displacement equation is (origin of the wave-maker movement considered to be $x_{0}=0$):
        \begin{equation}
            x_{s}(t)=e(t)=\frac{H}{k}
            \frac{\tanh(kc(t-T_{f}))}
            {d+H\left[1-\tanh^{2}(kc(t-T_{f}))\right]},
        \end{equation}

        where wave celerity is:
        \begin{equation}
            c=\sqrt{g(H+d)},
        \end{equation}

        generation time:
        \begin{equation}
            T_{f} = \frac{2}{kc}
            \left(
            3.8+\frac{H}{d}
            \right),
        \end{equation}

        outskirt coefficient (describes the way free-surface elevation tends towards the mean level at infinity):
        \begin{equation}
            k=\sqrt{\frac{3H}{4d^{2}(H+d)}}.
        \end{equation}

        Theoretical free-surface elevation will be as follows:
        \begin{equation}
            \eta(x_{s},t)=H \sech^{2}
            \left[
            k \left(
            c\left(t - \frac{T_{f}}{2} \right)
            +
            2\sqrt{\frac{H(H+d)}{3}} - x_{s}
            \right)
            \right].
        \end{equation}
    \end{itemize}
    Methods piston-based methods as well as wall disappear require ParticleParams to provide next options:
    \begin{enumerate}
        \item \verb|nwm_particles_start|,
        \item \verb|nwm_particles_end|.
    \end{enumerate}

    \item \verb|nwm_time_start|. Simulation time when NWM starts to generate waves.
    
    Optional parameter.
    \item \verb|nwm_wave_length|. Wave length $L$ to be generated.
    
    Mandatory when \verb|nwm| is \verb|NWM_METHOD_DYNAMIC|.
    \item \verb|nwm_wave_magnitude|. Wave height $H$ to be generated.
    
    Mandatory when \verb|nwm| is \verb|NWM_METHOD_DYNAMIC| or \verb|NWM_METHOD_SOLITARY_RAYLEIGH|.
\end{itemize}

\subsubsection{Output Control}
\begin{itemize}
    \item \verb|save_time|. Program will produce output for further processing every \verb|save_time| seconds of simulation. See 'data' directory of experiment.
    \item \verb|save_velocity|. Enable velocity saving in output.
    \item \verb|save_pressure|. Enable pressure saving in output.
    \item \verb|save_density|. Enable density saving in output.
\end{itemize}

\subsubsection{Dump Control}
\begin{itemize}
    \item \verb|use_dump|. Enable dump dump creation.
    \item \verb|dump_time|. Program will produce dump every \verb|dump_time| seconds of simulation. You can use that point in time to start from later. See 'dump' directory of experiment.
\end{itemize}

\subsubsection{Time Estimation}
\begin{itemize}
    \item \verb|use_custom_time_estimate_step|. Enables user-set time estimation step. Program will print amount of time left for simulation to finish every $k$ steps. If not set, program will print time estimation with every output.
    \item \verb|step_time_estimate|. Term $k$ in previous statement.
\end{itemize}

\subsubsection{Consistency Control}
\begin{itemize}
    \item \verb|consistency_check|. Enable consistency control: check if variables are NaN or infinite; check particles are outside of simulation domain.
    \item \verb|consistency_treatment|. Selects method for consistency control. Possible values of enumeration:
    \begin{itemize}
        \item 0: \verb|CONSISTENCY_PRINT|. Prints warning message on inconsistent value of variables and continues simulation. Simulation with such values of variables is undefined.
        \item 1: \verb|CONSISTENCY_STOP|. Prints error message and stop simulation process.
        \item 2: \verb|CONSISTENCY_FIX|. Fixes particles leaving simulation domain (mark them as non-existing). Prints error message and stop simulation process on NaN or infinite values of variables.
    \end{itemize}
\end{itemize}

\subsubsection{Optimization}
\begin{itemize}
    \item \verb|max_neighbours|. Maximum number of neighbours for particle. Affects on allocated memory for grid of neighbours.
    \item \verb|local_threads|. Local threads number for OpenCL kernels and OpenMP parallel sections.
\end{itemize}
\subsection{Particle Params}

\subsubsection{Target version}

\begin{itemize}
    \item \verb|params_target_version_major|
    \item \verb|params_target_version_minor|
    \item \verb|params_target_version_patch|
\end{itemize}

\subsubsection{Particle count}

\begin{itemize}
    \item \verb|nfluid|. Fluid particles count.
    \item \verb|nvirt|. Boundary (virtual) particles count.
    \item \verb|ntotal|. Total particles count (fluid + boundary).
\end{itemize}

\subsubsection{Geometry}
These parameters describe simulation domain area.

\begin{itemize}
    \item \verb|x_mingeom|. Left side of the domain.
    \item \verb|x_maxgeom|. Right side of the domain.
    \item \verb|y_mingeom|. Bottom of the domain.
    \item \verb|y_maxgeom|. Top of the domain.
\end{itemize}

\subsubsection{Particles parameters}

\begin{itemize}
    \item \verb|delta|. Initial distance between particles $\delta_{0}$.
    \item \verb|rho0|. Reference density of particles $\rho_{0}$.
    
    Optional parameter. Default value is $\rho_{0} = 1000$.
\end{itemize}

\subsubsection{Extra parameters}

\begin{itemize}
    \item \verb|depth|. Reference depth of the fluid $d$.
    
    Optional parameter. By default $d = \max\limits_{i}y_{i}$ on the start.
\end{itemize}

\subsubsection{Numerical Waves Maker}
If \verb|nwm| is piston or disappearing wall, affected boundary particles are from the contiguous block characterized by two indices: [start; end).

\begin{itemize}
    \item \verb|nwm_particles_start|. Index of the first particle in moving or disappearing block.
    \item \verb|nwm_particles_end|. Index of the next to the last particle in moving or disappearing block.
\end{itemize}
\subsection{Computing Params}

\subsubsection{Software version}
\begin{itemize}
    \item \verb|SPH2D_version_major|
    \item \verb|SPH2D_version_minor|
    \item \verb|SPH2D_version_patch|
\end{itemize}

\subsubsection{Params module version}
\begin{itemize}
    \item \verb|params_version_major|
    \item \verb|params_version_minor|
    \item \verb|params_version_patch|
\end{itemize}

\subsubsection{Common module version}
\begin{itemize}
    \item \verb|SPH2D_common_version_major|
    \item \verb|SPH2D_common_version_minor|
    \item \verb|SPH2D_common_version_patch|
\end{itemize}

\subsubsection{Computing module version}
\begin{itemize}
    \item \verb|SPH2D_specific_version_major|
    \item \verb|SPH2D_specific_version_minor|
    \item \verb|SPH2D_specific_version_patch|
    \item \verb|SPH2D_specific_version_name|
\end{itemize}

\subsubsection{Computed constants}
\begin{itemize}
    \item \verb|mass|. Mass of the particle $m$:
    \begin{equation}
        m = \rho_{0} \delta_{0}^{\text{dim}}.
    \end{equation}

    \item \verb|hsml|. Smoothing length $h$:
    \begin{equation}
        h = \delta_{0} h_{k}.
    \end{equation}

    \item \verb|cell_scale_k|. Size of grid cell in terms of $h$. Equals to 3 if any kernel is gauss. Equals to 2 otherwise.

    \item \verb|maxn|. Maximum count of particles in simulation:
    \begin{equation}
        \text{maxn} = 2^{1 + 
        \lfloor\log_{2}\text{ntotal}\rfloor}
    \end{equation}

    \item \verb|max_cells|. Count of grid cells to be processed.

    \item \verb|start_simulation_time|
\end{itemize}

\subsubsection{Numerical Waves Maker}
\begin{itemize}
    \item \verb|nwm_wave_number|. Generated wave number $k$.
    \item \verb|nwm_freq|. Generated wave angular frequency $\omega$.
    \item \verb|nwm_piston_magnitude|. If nwm is dynamic, piston magnitude $S_{0}$ computed by eq.~\ref{eq:pistonS0}.
\end{itemize}

\subsubsection{Particle type}\label{sec:parttype}
\begin{itemize}
    \item \verb|TYPE_WATER|. Fluid particles type identifier in output.
    \item \verb|TYPE_BOUNDARY|. Water particles type identifier in output.
    \item \verb|TYPE_NON_EXISTENT|. Identifier of particles that are not exist.
\end{itemize}
\subsection{Loading Params}\

It could be useful to load just a part of output data.
%
Any post processing program will load files with respect to 'LoadingParams.json'. 
%
Computing programs can't start experiment where 'LoadingParams.json' exists.

\begin{itemize}
    \item \verb|every_layers|. Load every $n^{\text{th}}$ layer.

    Optional parameter. Default value is 1.
    \item \verb|from|. Skip all layers before $t_{\text{from}}$ second. 

    Optional parameter. Default value is 0.
    
    \item \verb|to|. Skip all layers after $t_{\text{to}}$ second.

    Optional parameter. Default value is $t_{\text{max}}$.
\end{itemize}
\subsection{Height Testing Params}

\subsubsection{Common part}
\begin{itemize}
    \item \verb|mode|. Mode of height testing. Possible values are: 
    \begin{itemize}
        \item "space",
        \item "time".
    \end{itemize}
    
    \item \verb|y0|. $y_{0}$ in eq. \ref{eq:sp0} and \ref{eq:tp0}.
    
    Optional parameter. Default value is 0.
    \item \verb|y_k|. $y_{k}$ in eq. \ref{eq:sp0} and \ref{eq:tp0}.
    \item \verb|search_n|. Smoothing factor $s_{n}$: for a given point $x_{i}$ program will search particles with $\max y$ in interval $[x_{i} - hs_{n}; x_{i} + hs_{n}]$, where $h$ is smoothing length.
    \item \verb|particles_type|. Type of particles to consider. See section~\ref{sec:parttype}.

    Optional parameter. Default value is null (all particles are considered).
\end{itemize}

\subsubsection{Space profile}\

Calculate space profile ($y$ values) for a given points of time:
\begin{equation}
    y_{i} = \max\limits_{i}(y([x_{i} - hs_{n}; x_{i} + hs_{n}])),
\end{equation}
\begin{equation}
    i = \lceil \frac{1}{\delta_{0}} (\verb|x_maxgeom| - \verb|x_mingeom|) \rceil.
\end{equation}

Space profile points are transformed by coefficients and terms:
\begin{equation}\label{eq:sp0}
    \begin{cases}
      \begin{tabular}{l}
          $\Bar{x} = x_{k} (x - x_{0})$, 
          \\
          $\Bar{y} = y_{k} (y - y_{0})$.
      \end{tabular}  
    \end{cases} 
\end{equation}

\begin{itemize}
    \item \verb|t|. Array of points in time to calculate time profile. Each point will be separate output file.
    \item \verb|x0|. $x_{0}$ in eq.~\ref{eq:sp0}.
    
    Optional parameter. Default value is 0.
    \item \verb|x_k|. $x_{k}$ in eq.~\ref{eq:sp0}.
    
    Optional parameter. Default value is 1.
\end{itemize}

\subsubsection{Time profile}\

Calculate time profile ($y$ values) for a given points in space:
\begin{equation}
    y_{i} = \max\limits_{i}(y([x - hs_{n}; x + hs_{n}])),
\end{equation}
\begin{equation}
    i = \lceil \frac{1}{\Delta t} (\verb|simulation_time| - \verb|save_time|) \rceil.
\end{equation}

Time profile points are transformed by coefficients and terms:
\begin{equation}\label{eq:tp0}
    \begin{cases}
      \begin{tabular}{l}
          $\Bar{t} = t_{k} (t - t_{0})$, 
          \\
          $\Bar{y} = y_{k} (y - y_{0})$.
      \end{tabular}  
    \end{cases} 
\end{equation}

\begin{itemize}
    \item \verb|x|. Array of points in space to calculate time profile. Each point will be separate output file.
    
    \item \verb|t0|. $t_{0}$ in eq.~\ref{eq:tp0}.
    
    Optional parameter. Default value is 0.
    \item \verb|t_k|. $t_{k}$ in eq.~\ref{eq:tp0}.

    Optional parameter. Default value is 1.
\end{itemize}
\end{document}
