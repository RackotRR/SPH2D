
\subsection{Model Params}

\subsubsection{Target version}

\begin{itemize}
    \item \verb|params_target_version_major|
    \item \verb|params_target_version_minor|
    \item \verb|params_target_version_patch|
\end{itemize}

\subsubsection{Smoothing kernel function}
\label{sec:skf}

Possible values:

\begin{itemize}
    \item 1: \verb|SKF_CUBIC|.
    \begin{equation}
        W(q) = \kappa_{\text{cubic}}
        \begin{cases}
          \begin{tabular}{l l}
              $\frac{2}{3} - q^{2} + \frac{1}{2} q^{3}$
              &
              $q \leq 1$, 
              
              \\
              
              $\frac{1}{6} \left(2 - q\right)^{3}$
              &
              $1 < q \leq 2$,
              \\
              
              0
              &
              $q > 2$.
          \end{tabular}  
        \end{cases} 
    \end{equation}

    \item 2: \verb|SKF_GAUSS|.
    \begin{equation}
        W(q) = \kappa_{\text{gauss}}
        \begin{cases}
          \begin{tabular}{l l}
              $\displaystyle{\frac{1}{h^{2}\pi}e^{q^{-2}}}$
              &
              $q \leq 3$, 
              
              \\
              
              0
              &
              $q > 3$.
          \end{tabular}  
        \end{cases} 
    \end{equation}

    \item 3: \verb|SKF_WENDLAND|.
    \begin{equation}
        W(q) = \kappa_{\text{wendland}}
        \begin{cases}
          \begin{tabular}{l l}
              $\displaystyle{\left(
              1 - \frac{q}{2}
              \right)^4
              (2q + 1)}$
              &
              $q \leq 2$, 
              
              \\
              
              0
              &
              $q > 2$.
          \end{tabular}  
        \end{cases} 
    \end{equation}
    
    \item 3: \verb|SKF_DESBRUN|.
    \begin{equation}
        W(q) = \kappa_{\text{desbrun}}
        \begin{cases}
          \begin{tabular}{l l}
              $\displaystyle{\left(2 - q
              \right)^3}$
              &
              $q \leq 2$, 
              
              \\
              
              0
              &
              $q > 2$.
          \end{tabular}  
        \end{cases} 
    \end{equation}
\end{itemize}


\subsubsection{Density}

\begin{itemize}
    \item \verb|density_treatment|. Density integration function. Possible values of enumeration:
    \begin{itemize}
        \item 0: \verb|DENSITY_SUMMATION|. Use SPH summation over kernel:
        \begin{equation}
            \rho_{j} = \sum\limits_{i} m_{i} W_{ji}.
        \end{equation}

        \item 1: \verb|DENSITY_CONTINUITY|. Use integration of continuity equation:
        \begin{equation}
            \frac{D\rho_{j}}{Dt} = \sum\limits_{i} m_{i} \vec{v_{ji}} \cdot \vec{\nabla}_{j} W_{ji}.
        \end{equation}

        \item 2: \verb|DENSITY_CONTINUITY_DELTA|. Use integration of continuity equation with density diffussion:
        \begin{equation}\label{eq:CSPM}
             \frac{D\rho_{j}}{Dt} = \sum\limits_{i} m_{i} \vec{v}_{ji} \cdot \vec{\nabla}_{j} W_{ji} + 
             2\delta_{\text{density}} h c
             \sum\limits_{i}m_{i} \frac{(\rho_{i} - \rho_{j})}{\rho_{i}} \frac{\vec{r}_{ji}}{|\vec{r}_{ji}|^2} \cdot \vec{\nabla}_{j} W_{ji}.
        \end{equation}
    \end{itemize}

    \item \verb|density_normalization|. Enable boundary deficiency correction:
    \begin{equation}
        \rho_{j} = \frac{
        \sum\limits_{i}m_{i}W_{ji}
        }{
        \sum\limits_{i}
        \left(\frac{m_{i}}{\rho_{i}}\right) W_{ji}
        }
    \end{equation}

    Optional. Available if \verb|density_treatment == 0|.
    \item \verb|density_skf|. See section~\ref{sec:skf}.
    \item \verb|density_delta_sph_coef|. $\delta_{\text{density}}$ parameter in eq.~\ref{eq:CSPM}.
    
    Mandatory if \verb|density_treatment == 2|.
\end{itemize}

\subsubsection{Equation of state}

\begin{itemize}
    \item \verb|eos_sound_vel_method|. Method for sound velocity choosing. Possible values of enumeration:

    \begin{itemize}
        \item 0: \verb|EOS_SOUND_VEL_DAM_BREAK|. Use dam break assumption:
        \begin{equation}\label{eq:cdam}
            c = \sqrt{200gdc_{k}}.
        \end{equation}

        \item 1: \verb|EOS_SOUND_VEL_SPECIFIC|. Use custom sound velocity.
        \begin{equation}\label{eq:ccustom}
            c = c_{\text{user}}.
        \end{equation}
    \end{itemize}

    \item  \verb|eos_sound_vel_coef|. $c_{k}$ parameter in eq.~\ref{eq:cdam}.

    Mandatory if \verb|eos_sound_vel_method == 0|.
    
    \item \verb|eos_sound_vel|. User-provided sound velocity $c_{\text{user}}$ in eq.~\ref{eq:ccustom}.
    
    Mandatory if \verb|eos_sound_vel_method == 1|.
\end{itemize}

\subsubsection{Internal Forces}

\begin{itemize}
    \item \verb|intf_sph_approximation|. SPH momentum equation form. Possible values of enumeration:
    \begin{itemize}
        \item 1: \verb|INTF_SPH_APPROXIMATION_1|. SPH momentum equation form:
        \begin{equation}
            \frac{D\vec{v}_j}{D t} = 
            -\sum\limits_{i} m_{i} 
            \left(
            \frac{p_{i} + p_{j}}{\rho_{i} \rho_{j}}
            \right) \vec{\nabla_{j}} W_{ji}.
        \end{equation}
        
        \item 2: \verb|INTF_SPH_APPROXIMATION_2|. SPH momentum equation form:
        \begin{equation}
            \frac{D\vec{v}_j}{D t} = 
            -\sum\limits_{i} m_{i} 
            \left(
            \frac{p_{j}}{\rho_{j}^{2}} + 
            \frac{p_{i}}{\rho_{i}^{2}}
            \right) \vec{\nabla_{j}} W_{ji}.
        \end{equation}
    \end{itemize}

    \item \verb|intf_hsml_coef|. $h_{k}$ in smoothing kernel length equation:
    \begin{equation}
        h = \delta_{0} \cdot h_{k},
    \end{equation}
    where $\delta_{0}$ is initial distance between particles.

    \item \verb|intf_skf|. See section~\ref{sec:skf}.
\end{itemize}

\subsubsection{Artificial Pressure}

\begin{itemize}
    \item \verb|artificial_pressure|. Enable additional factor in momentum equation for tensile instability correction:
    \begin{equation}\label{eq:artpress0}
        \frac{D\vec{v}_j}{D t} = 
        -\sum\limits_{i} m_{i} 
        \left(
        \frac{p_{j}}{\rho_{j}^{2}} + 
        \frac{p_{i}}{\rho_{i}^{2}} +
        R\Tilde{W}_{ji}^{n}
        \right) \vec{\nabla_{j}} W_{ji},
    \end{equation}
    where $R\Tilde{W}_{ji}^{n}$ is artificial pressure term. SPH approximation function could be of both types.
    
    \begin{equation}
        R = R_{i} + R_{j},
    \end{equation}

    where any indexed $R_{x}$ is:
    \begin{equation}\label{eq:artpress1}
        R_{x} = 
        \begin{cases}
          \begin{tabular}{l l}
              $\displaystyle{
              \frac{\epsilon|p_{x}|}{\rho_{x}^{2}}
              }$
              &
              $p_{x} < 0$, 
              
              \\
              
              0
              &
              $p_{x} \geq 0$,
          \end{tabular}  
        \end{cases} 
    \end{equation}
    where $\epsilon$ is artificial pressure coefficient.
    
    \begin{equation}
        \Tilde{W}_{ji} = \frac{W(\vec{r}_{ji}, h)}{W(\delta_{0}, h)},
    \end{equation}
    where $\delta_{0}$ is initial distance between particles.

    \item \verb|artificial_pressure_skf|. See section \ref{sec:skf}.
    \item \verb|artificial_pressure_index|. Term $n$ in eq.~\ref{eq:artpress0}.
    \item \verb|artificial_pressure_coef|. Term $\epsilon$ in eq.~\ref{eq:artpress1}.
\end{itemize}

\subsubsection{Artificial Viscosity}
\begin{itemize}
    \item \verb|artificial_viscosity|. Enable additional factor in momentum equation for stability correction:
    \begin{equation}
        \frac{D\vec{v}_j}{D t} = 
        -\sum\limits_{i} m_{i} 
        \left(
        \frac{p_{j}}{\rho_{j}^{2}} + 
        \frac{p_{i}}{\rho_{i}^{2}} +
        \Pi_{ji}
        \right) \vec{\nabla_{j}} W_{ji},
    \end{equation}
    where $\Pi_{ji}$ is artificial viscosity term. SPH approximation function could be of both types.
    $\Pi_{ji}$ is:
    \begin{equation}\label{eq:artvisc}
        \Pi_{ji} = 
        \begin{cases}
          \begin{tabular}{l l}
              $\displaystyle{
              \frac{-\alpha_{\Pi}c_{ji}\phi_{ji} + \beta_{\Pi}\phi_{ji}^{2}}{\rho_{ji}^{2}}
              }$
              &
              $\vec{v}_{ji} \cdot \vec{r}_{ji} < 0$, 
              
              \\
              
              0
              &
              $\vec{v}_{ji} \cdot \vec{r}_{ji} \geq 0$,
          \end{tabular}  
        \end{cases} 
    \end{equation}

    \begin{equation}
        \phi_{ji} = 
        \frac
        {h_{ji}\vec{v}_{ji} \cdot \vec{r}_{ji}}
        {|\vec{r}_{ji}|^2 + (\epsilon_{\Pi} h_{ji})^2 },
    \end{equation}

    \begin{equation}
        \begin{tabular}{l l l}
              $\displaystyle{
              \rho_{ji} = \frac{\rho_{j} + \rho_{i}}{2}
              }$,
              &
              $\displaystyle{
              c_{ji} = \frac{c_{j} + c_{i}}{2}
              }$, 
              &
              $\displaystyle{
              h_{ji} = \frac{h_{j} + h_{i}}{2}
              }$,
          \end{tabular}  
    \end{equation}
    
    \begin{equation}
        \begin{tabular}{l l}
              $\displaystyle{
              \vec{v}_{ji} = \vec{v}_{j} - \vec{v}_{i}
              }$,
              &
              $\displaystyle{
              \vec{r}_{ji} = \vec{r}_{j} - \vec{r}_{i}
              }$.
          \end{tabular}  
    \end{equation}

    \item \verb|artificial_viscosity_skf|. See section~\ref{sec:skf}.
    \item \verb|artificial_shear_visc|. Term $\alpha_{\Pi}$ in eq.~\ref{eq:artvisc}.
    \item \verb|artificial_bulk_visc|. Term $\beta_{\Pi}$ in eq.~\ref{eq:artvisc}.
\end{itemize}

\subsubsection{Average Velocity}
\begin{itemize}
    \item \verb|average_velocity|. Enable XSPH velocity smoothing factor:
    \begin{equation}
        \frac{D\vec{r}_j}{Dt} = \vec{v}_{j} + \vec{v}_{j\;\text{avg}},
    \end{equation}
    where $v_{avg}$ is:
    \begin{equation}\label{eq:xsph}
        \vec{v}_{j\;\text{avg}} = -\varepsilon_{\text{avg}}
        \sum\limits_{i}\frac{m_{i}}{\rho_{i}}\vec{v}_{ji}W_{ji}.
    \end{equation}

    \item \verb|average_velocity_skf|. See section~\ref{sec:skf}.
    \item \verb|average_velocity_coef|. Term $\varepsilon_{\text{avg}}$ in eq.~\ref{eq:xsph}.
\end{itemize}

\subsubsection{Time Integration}
\begin{itemize}
    \item \verb|simulation_time|. Total simulation time to stop experiment.
    \item \verb|dt_correction_method|. Method of $\Delta t$ choosing. Possible values of enumeration:
    \begin{itemize}
        \item 0: \verb|DT_CORRECTION_CONST_VALUE|. Use user-provided constant value.
        \begin{equation}\label{eq:dtcvalue}
            \Delta t_{\tau} = \Delta t_{\text{user}}.
        \end{equation}
        \item 1: \verb|DT_CORRECTION_CONST_CFL|. $\Delta t$ is computed before experiment start with CFL condition:
        \begin{equation}\label{eq:dtccfl}
            \Delta t_{\tau} = \Delta t_{0} = 
            \text{CFL} \frac{h}{c (1 + 1.2\alpha_{\Pi})}
        \end{equation}
        \item 2: \verb|DT_CORRECTION_DYNAMIC|. $\Delta t$ is computed on every step with CFL condition:
        \begin{equation}\label{eq:dtdcfl}
            \Delta t_{\tau} = \text{CFL}\min(\Delta t_{f}, \Delta t_{\phi}),
        \end{equation}
        \begin{equation}
            \Delta t_{f} = \min\limits_{i}
            \left(
            \sqrt{\frac{h}{a_{i}}}
            \right),
        \end{equation}
        \begin{equation}
            \Delta t_{\phi} = \min\limits_{i}
            \left(
            \frac{h}{c + \phi}
            \right).
        \end{equation}
    \end{itemize}

    \item \verb|dt|. Term $\Delta t_{\text{user}}$ in eq.~\ref{eq:dtcvalue}.
    \item \verb|CFL_coef|. Term CFL in eq.~\ref{eq:dtccfl} and eq.~\ref{eq:dtdcfl}.
\end{itemize}

\subsubsection{Boundary Treatment}
\begin{itemize}
    \item \verb|boundary_treatment|. Selects method for boundaries. Possible values of enumeration:
    \begin{itemize}
        \item 0: \verb|SBT_DYNAMIC|. Boundary particles are the same as fluid but have constant positions.
        \item 1: \verb|SBT_REPULSIVE|. Boundary particles are dynamic with additional term in momentum equation:
        
        \begin{equation}
        \frac{D\vec{v}_j}{D t} = 
        -\sum\limits_{i} 
        \left[
        m_{i} 
        \left(
        \frac{p_{j}}{\rho_{j}^{2}} + 
        \frac{p_{i}}{\rho_{i}^{2}}
        \right) \vec{\nabla_{j}} W_{ji}
        + \vec{\Upsilon}_{ji}
        \right],
        \end{equation}
        
        where $\vec{\Upsilon}_{ji}$ is computed if $j$ is fluid particle and $i$ is boundary particle:
        
        \begin{equation}
        \vec{\Upsilon}_{ji} = 
        \frac{\vec{r}_{ji}}{|\vec{r}_{ji}|}
        \begin{cases}
          \begin{tabular}{l l}
              $\displaystyle{
              D\left[
              \left(
                \frac{r_{0}}{|\vec{r}_{ji}|}
              \right)^{\alpha_{1}}
              -
              \left(
                \frac{r_{0}}{|\vec{r}_{ji}|}
              \right)^{\alpha_{2}}
              \right]
              }$
              &
              $|\vec{r_{ji}}| \leq r_{0}$, 
              
              \\
              
              0
              &
              $|\vec{r_{ji}}| > r_{0}$.
          \end{tabular}  
        \end{cases} 
        \end{equation}
        Equation parameters are considered built-in:
        \begin{equation}
          \begin{tabular}{l}
            $r_{0} = 2h$,
            \\
            $D = 5gd$,
            \\
            $\alpha_{1} = 12$,
            \\
            $\alpha_{2} = 4$,
          \end{tabular}  
        \end{equation}
        where $d$ is initial water depth.
    \end{itemize}
\end{itemize}

\subsubsection{Numerical Waves Maker}
\begin{itemize}
    \item \verb|nwm|. Waves maker method. Possible values of enumeration:
    \begin{itemize}
        \item 0: \verb|NWM_NO_WAVES|. No waves generator.
        \item 2: \verb|NWM_METHOD_DYNAMIC|. First order waves generator. Piston-type wavemaker. 

        Makes waves with surface displacement of type:
        \begin{equation}
            \eta(x, t) = \frac{H}{2}cos(\omega t - kx + \delta),
        \end{equation}
        where $H$ is wave height, $x$ is distance and $\delta$ is the initial phase.
        $\omega = 2\pi/T$ is the angular frequency and $k=2\pi/L$ is the wave number with $T$ equal to the wave period and $L$ the wave length.

        Piston displacement equation is:
        \begin{equation}
            e_{1}(t) = \frac{S_{0}}{2} sin(\omega t + \delta),
        \end{equation}
        where $S_{0}$ is piston magnitude:
        \begin{equation}\label{eq:pistonS0}
            S_{0} = H \frac
            {\sinh{kd}\cosh{kd} + kd}
            {2\sinh{kd}^2},
        \end{equation}
        where $d$ is depth.
        
        \item 4: \verb|NWM_METHOD_WALL_DISAPPEAR|.
    \end{itemize}
    Methods \verb|NWM_METHOD_DYNAMIC| and \verb|NWM_METHOD_WALL_DISAPPEAR| require ParticleParams to provide\\ \verb|nwm_particles_start| and \verb|nwm_particles_end|.

    \item \verb|nwm_time_start|. Simulation time when NWM starts to generate waves.
    
    Optional parameter.
    \item \verb|nwm_wave_length|. Wave length $L$ to be generated.
    
    Mandatory when \verb|nwm == NWM_METHOD_DYNAMIC|.
    \item \verb|nwm_wave_magnitude|. Wave height $H$ to be generated.
    
    Mandatory when \verb|nwm == NWM_METHOD_DYNAMIC|.
\end{itemize}

\subsubsection{Output Control}
\begin{itemize}
    \item \verb|save_time|. Program will produce output for further processing every \verb|save_time| seconds of simulation. See 'data' directory of experiment.
    \item \verb|save_velocity|. Enable velocity saving in output.
    \item \verb|save_pressure|. Enable pressure saving in output.
    \item \verb|save_density|. Enable density saving in output.
\end{itemize}

\subsubsection{Dump Control}
\begin{itemize}
    \item \verb|use_dump|. Enable dump dump creation.
    \item \verb|dump_time|. Program will produce dump every \verb|dump_time| seconds of simulation. You can use that point in time to start from later. See 'dump' directory of experiment.
\end{itemize}

\subsubsection{Time Estimation}
\begin{itemize}
    \item \verb|use_custom_time_estimate_step|. Enables user-set time estimation step. Program will print amount of time left for simulation to finish every $k$ steps. If not set, program will print time estimation with every output.
    \item \verb|step_time_estimate|. Term $k$ in previous statement.
\end{itemize}

\subsubsection{Consistency Control}
\begin{itemize}
    \item \verb|consistency_check|. Enable consistency control: check if variables are NaN or infinite; check particles are outside of simulation domain.
    \item \verb|consistency_treatment|. Selects method for consistency control. Possible values of enumeration:
    \begin{itemize}
        \item 0: \verb|CONSISTENCY_PRINT|. Prints warning message on inconsistent value of variables and continues simulation. Simulation with such values of variables is undefined.
        \item 1: \verb|CONSISTENCY_STOP|. Prints error message and stop simulation process.
        \item 2: \verb|CONSISTENCY_FIX|. Fixes particles leaving simulation domain (mark them as non-existing). Prints error message and stop simulation process on NaN or infinite values of variables.
    \end{itemize}
\end{itemize}

\subsubsection{Optimization}
\begin{itemize}
    \item \verb|max_neighbours|. Maximum number of neighbours for particle. Affects on allocated memory for grid of neighbours.
    \item \verb|local_threads|. Local threads number for OpenCL kernels and OpenMP parallel sections.
\end{itemize}